\section{Target System}

Run Group F will use a variety of gas targets (hydrogen, deuterium, and helium) at room temperature and
at 1 to 7 atm absolute pressure (7 atm = 88 psig). Nitrogen gas at 1 atm will also be used for purging the gas target.
The gas will be contained within a thin-walled (50 $\mu$m wall thickness) polyimide (Kapton) tube (straw), which will be 
450 mm long and 6 mm in diameter. 
The beam entrance and exit windows will be made out of 15 $\mu$m thick aluminum foil (2014 alloy).
The target will be mounted inside the centre bore of the RTPC. The target beam exit window will protrude 
20 mm downstream of the RTPC and will be enshrouded with a polyimide capsule containing inert helium gas at 1 atm.
The same helium buffer gas will also fill the radial space between the gas target and the inner window of the RTPC.

The target gas is supplied at the upstream end via a capillary copper tube from a 3 liter buffer volume, 
separated by a remote-controlled isolation valve. The gas handling system upstream of the isolation valve 
complies with ASME B31.3 process piping code.


WE NEED DETAILED DESCRIPTION OF THE GAS SYSTEM, entry and exit windows, and integration
into CLAS12.

%The experiment will use both solid ($^{12}$C) and cryogenic (LH$_2$) targets. The solid 
%target, a 0.5-mm-thick carbon wire, will be mounted on the fork of the 2H01 wire harp. The 
%wire harp will be mounted at the nominal location of the CLAS12 target. The fork with wires 
%(30~mm spacing between arms of the fork) is mounted on a shaft connected to a stepper 
%motor motion system. The system is aligned in a way that the beam is centered between the 
%fork arms, which allows the target to be moved in and out of the beam without interference 
%of the support frames.
%   
%The cryogenic target system is a joint product of Saclay (France) and JLab (the same target 
%system that has worked for 15~years with the CLAS detector). The target cell is a $\sim 20$~mm 
%diameter, 5-cm-long Kapton tube with $30~\mu$m entrance and exit windows. There is an 
%insulation made of a 65~$\mu$m-thick-aluminized Mylar around the target cell. The cell is 
%connected to the condensing system with three $\sim$1.5-m-long Aluminum pipes. The whole system is 
%in the vacuum. The target cell is inside a Rohacell scattering chamber, with a wall thickness 
%of 10~mm, that is connected to the upstream beampipe. The scattering chamber has a 51~cm long,
%25~mm diameter carbon extension pipe, with a wall thickness of 1~mm. The thickness of the 
%aluminum exit window on the extension tube is 50~$\mu$m.
               
\subsection{Hazards} 

The main hazard is a potential rupture of the target cell or the target entrance and exit windows. 
If the target is filled with $^4$He at the time, the only hazard would be a loud noise and potential damage
to the RTPC detector surrounding the target due to a pressure shock. If the target is filled with
$^1$H or $^2$H, an additional hazard will be the presence of these flammable gases within the RTPC
environment. The total gas volumes involved are too small to create a ODH risk.

%The LH$_2$ target contains a condensed cryogenic fluid and is considered a pressure vessel. 
%Sudden warming of the target due to a vacuum breach could result in rapid expansion of the 
%target fluid. The system is designed to safely vent the excess pressure unless the vent 
%lines are blocked by frozen hydrogen and/or frozen contaminates in the gas. Failure of the 
%foam, aluminum extension, or the thin window of the scattering chamber could produce a loud 
%noise and could result in a failure of the target integrity. The target utilizes flammable 
%gas (hydrogen) during operation. Failure of the system could release flammable gas into the 
%hall. The hydrogen target gas and the helium used in the target refrigerator are potential 
%ODH risks, and failure of either system could reduce the oxygen levels in the hall.
%
%There are no hazards associated with moving the solid target into the beam. The stepping 
%motor linear actuator will be operated using EPICS controls. The GUI for operation of the 
%target will have preset coordinates. 

\subsection{Mitigations}

The design and construction of the gas target is in accordance to ASME B31.3 process piping code,
except downstream of the isolation valve. This section includes the target cell and is qualified by equivalent measures.
These include limiting the amount of flammable gas in this section and preventing excess flow into this 
section from the buffer volume in case of a sudden pressure loss due to a rapture of the target cell or its windows. 
A flame arrestor is preventing the ignition of flammable gas upstream of this section (upstream of the isolation valve).

During operation the target tube and the thin entrance and exit windows are surrounded by the Hall~B 
CLAS12 Central Detectors and are therefore difficult to access. 
The target will only be filled with gas in excess of 1 atm pressure when installed inside the CLAS12 Central Detector.

Two H$_2$/D$_2$ gas detectors are installed near the target location, one above the right hand 
side of the electronics rack and another above the cryostat. In case of a detected leak, the 
control system will immediately shut off the supply of gas to the target.

A pressure sensor is mounted downstream of the isolation valve to detect a sudden loss of pressure and
immediately shut off the gas supply.

The quantity of flammable gas (H$_2$ or D$_2$) downstream of the isolation valve is less than 1000 ft lbs in energy (1355 J).
The ratio of energy per volume for hydrogen or deuterium is 14.5 J/cm$^3$. This includes chemical heat of combustion 
(13 J/cm$^3$) and mechanical explosion energy (1.5 J/cm$^3$, Brode equation). 
For the limit of 1355 J this corresponds to 13.3 cm$^3$ of deuterium (or hydrogen) gas at 7 atm (93.4 cm$^3$ at 1 atm).

This volume limit in the section qualified by equivalent measures will be achieved by limiting the volume
of the target cell plus the capillary supply tubing up to the isolation valve to 13.3 cm$^3$. To limit the amount of 
gas flow into the target cell, the supply tubing inner diameter will be capillary and an excess flow valve 
will limit the flow in case of sudden loss of pressure in the target cell due to bursting.

\begin{itemize}
\item The area shall be posted ``Danger Flammable Gases.  No Ignition Sources",

\item Combustibles and ignition sources shall be minimized within 10~ft or 3~m of target's gas 
handling equipment and piping.
\end{itemize}

The target does not operate in a confined space, and the total quantity of hydrogen/helium in 
the system is under 1000 standard liters. This presents a negligible oxygen deficiency risk in 
Hall~B and therefore is a class-0 ODH installation.

Hydrogen shall be loaded into the system by qualified personnel only, and those personnel shall 
follow approved operational gas handling procedures.   Upon loss of target gas pressure, the 
control system shall automatically shut off the gas supply. 

The target control software includes numerous alarms (temperature, pressure, vacuum, etc.) 
to alert users to potential problems. 

\subsection{Responsible Personnel}

The target system will be maintained by the Hall~B Engineering Group.  

\begin{table}[!htb]
\centering
\begin{tabular}{|c|c|c|c|c|}
\hline
 Name&Dept.&Phone&email&Comments \\ \hline
Engineering on call & Hall~B&(757)-748-5048&& 1st contact  \\ \hline
K. Bruhwel& Hall~B&x7868&\href{mailto:bruhwel@jlab.org}{\nolinkurl{bruhwel@jlab.org}}&2nd contact \\ \hline
D. Insley & Hall~B&x6071&\href{mailto:tilles@jlab.org}{\nolinkurl{tilles@jlab.org}}  &3rd contact \\ \hline
R. Miller &Hall~B&x7867&\href{mailto:rmiller@jlab.org}{\nolinkurl{rmiller@jlab.org}} &4th contact \\ \hline
\end{tabular}
\caption{Personnel responsible for the CLAS12 target system.} 
\label{tb:target}
\end{table}
