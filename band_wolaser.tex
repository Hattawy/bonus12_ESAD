\section{Backward Angle Neutron Detector}
The Backward Angle Neutron Detector (BAND) is placed at the top of the SVT cart upstream of the CLAS12 target. It consists of 116 scintillator bars, arranged in 18 rows and 5 layers. Four bars are missing in the bottom of the detector due to obstruction. The bars have a cross section of $7.2 \times 7.2\,\mathrm{cm}^{2}$ and they are 164 and $202\,\mathrm{cm}$ long in the upper region of BAND. In the bottom region the bars are divided into two shorter bars $51\,\mathrm{cm}$ to have a hole for the beam line and target installation. All bars are read-out on both sides by PMTs (Hamamatsu R7724 and ET9214) giving a total of 232 active channels. 

In front of the first active layer of BAND, a veto layer is installed with 24 bars read-out only on one side. Therefore, the total number of channels for BAND is 256.
The PMTs are placed in the fridge field region of the solenoid, and due to this they are encased in a cylindrical shielding made up by a 2-mm-thick layer of mu-metal.

In order to operate the PMTs, high voltages (typically in the range of 1500 V) are provided by a multi-channel CAEN SYS4527 mainframe with 11 A15350 cards (24 channel each).
The signal of each PMT is sent to an splitter. The splitter are UVA splitter previously used by other experiments at JLab.
From the splitter one signal is sent to flash-ADCs (250 VXS, 16 channels/board, made and owned by JLab) while the other signal is sent to  discriminators (16 channels/board).
The discriminated time signal then goes to a TDC (CAEN VX1190A, 128 channels/board, 100 ps/channel resolution). The read-out system is installed left of BAND in beam direction. 
In total, the system consists of 16 flash-ADCs in one VXS crate, 16 discriminators and a TDC in a VME crate and 16 splitters.  Furthermore, a signal distribution card for the flash-ADCs and trigger interface boards are installed in the crates.


\subsection{Hazards} 
\indent
\subsubsection{Electrical hazard}
The electrical hazard to personnel can come from the high voltage which powers the PMTs, which need about 1500 V to function. 

\subsubsection{Magnetic field hazard}
BAND is placed in the fringe field of the solenoid (50 - 100 gauss). A hazard may arise during maintenance operations in case metallic tools are used and for people with cardiac pacemakers, other electrical medical devices, or metallic implants.


\subsection{Mitigations}
\indent
\subsubsection{Electrical hazard mitigations} 
The maximum current provided by the HV distribution boards is quite low ($<1\,\mathrm{mA}$). All mechanical structures are properly grounded. The HV boards must not be accessed during operation; during maintenance work, performed by trained personnel, the HV is turned off, and the power supply is locked and tagged.
\subsubsection{Magnetic field hazard mitigations}
After all sort of maintenance work is done on BAND, the area must be inspected and all ferromagnetic tools must be removed, before the field of the solenoid is ramped up again. Also, before the field can be turned on the PMT housings and magnetic shields should be throughly inspected, to make sure that they are no loose parts. 



\subsection{Responsible personnel}
\indent

Individuals responsible for the system are:

\begin{table}[!htb]
 \centering
 \begin{tabular}{|c|c|c|c|c|}
\hline
 Name&Dept.&Phone&email&Comments \\ \hline
 Expert on call& &&& 1st contact \\ \hline
O. Hen & MIT & &\href{mailto:hen@mit.edu}{\nolinkurl{hen@mit.edu}}& Contact \\ \hline
L. Weinstein & ODU &  &\href{mailto:weinstein@jlab.org}{\nolinkurl{weinstein@jlab.org}}& Second contact  \\ \hline

 \end{tabular}
\caption{Personnel responsible for the Backward Angle Neutron Detector.} 
\label{tb:band}
\end{table}

