\section{Electromagnetic Calorimeter}

The CLAS12 Electromagnetic Calorimeter (EC) package includes both the legacy CLAS 
electromagnetic calorimeters (ECAL) and the new pre-shower calorimeter (PCAL) modules 
installed just upstream of ECAL. Both ECAL and PCAL are lead-scintillator sampling 
calorimeters consisting in total of $54$ layers of 1-cm-thick scintillator strips and 
$52$ layers of 2-mm-thick lead sheets. Photomultiplier tubes (PMTs) are used for light 
readout. The total number of readout channels is $2448$. The nominal operational voltages 
for the ECAL and PCAL PMTs are $2200$~V and $900$~V, respectively. 

\subsection{Hazards} 

Hazards associated with this device are electrical shock or damage to the PMTs if the 
housing is opened with the HV on and the PMTs are exposed to room light.  Access to some 
PMTs requires either ladders or manlift operations with potential fall hazards. Accessing 
signal cables below the floor gratings requires grating removal and poses a potential trip 
hazard over the open space. 

\subsection{Mitigations}

Whenever any work has to be done on the calorimeter PMTs, the HV must be turned off. If work 
has to be done on the CAEN HV power supply, i.e. replacing HV cards, the HV mainframe must 
be powered off using the rear power switch to disable all circuits. Both extension and step 
ladders must be secured to structural beams or rails when accessing PMTs, and a harness must 
be worn for manlift operations. Open floor gratings must be surrounded on both sides by 
warning cones or yellow rope.       

\subsection{Responsible Personnel}

Individuals responsible for the EC system are:

\begin{table}[!htb]
\centering
\begin{tabular}{|c|c|c|c|c|} \hline
Name&Dept.&Phone&email&Comments \\ \hline
Expert on call& &(757)-810-1489&& 1st contact \\ \hline
C. Smith &UVA/JLab&&\href{mailto:lcsmith@jlab.org}{\nolinkurl{lcsmith@jlab.org}}&2nd contact \\ \hline
\end{tabular}
\caption{Personnel responsible for the CLAS12 EC system.} 
\label{tb:ec}
\end{table}

